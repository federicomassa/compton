\documentclass[11pt]{article}
\usepackage{makeidx}
\usepackage{graphicx}
\usepackage{amsmath}
\usepackage{amssymb}
\usepackage{latexsym}
\renewcommand\refname{Referenze}
\usepackage[utf8x]{inputenc}
\usepackage{titlesec}
\usepackage{bm}
\usepackage{mathtools}
\usepackage[document]{ragged2e}
\titleformat{\section}{\huge\normalfont\bf}{\thesection.\hspace{5pt}}{5pt}{\vspace{1cm}}
\titleformat*{\subsection}{\Large\bfseries}
\usepackage[inner=3cm,outer=3cm]{geometry}

\makeindex

\begin{document}
\justify
\printindex
\Large{A.a. 2013-2014}
\vspace{10cm}
\begin{center}
\Huge\textbf{Diffusione Compton}
\end{center}

\vspace{2cm}
\begin{flushleft}
\textit{Gruppo \textsc{1}} \\
\medskip
Federico \textsc{Massa} \\ 
Marco \textsc{Montella}
\end{flushleft}



\newpage

\begin{abstract}
\justify
 

\end{abstract}
\bigskip

\section{Introduzione}



\section{Apparato sperimentale} \label{sec:apparato}
\subsection{Goniometro}



Essendo lo scintillatore inorganico vincolato a muoversi su un arco di circonferenza, risulta di fondamentale importanza la misura della posizione angolare di quest'ultimo rispetto a uno zero che non deve necessariamente coincidere con il centro geometrico del fascio. \\
A questo scopo si è posizionato lo scintillatore plastico a circa $2 cm$ dall'estremità del collimatore e nel punto medio della congiungente degli estremi della guida dell'inorganico. Abbiamo quindi posizionato un goniometro prestando attenzione al fatto che il centro si trovasse sulla verticale del punto di incontro tra l'asse del collimatore e il centro del plastico. \'E legittimo supporre infatti che la maggior parte delle interazioni avvengano in un intorno di questo punto, le cui dimensioni sono determinate dalla larghezza del fascio. \\

\section{Operazioni preliminari}

\subsection{Stima della dose}

\subsection{Determinazione del punto di lavoro degli scintillatori}
Prendendo in considerazione lo scintillatore plastico, si è in primo luogo tracciata la curva dei conteggi in funzione della tensione di alimentazione. \\
Osservando l'assenza di una qualsivoglia struttura di plateau, si è effettuata un'analisi separata del rumore riposizionando lo scintillatore plastico in una zona del tutto schermata dal fascio principale. \\ 
Alla tensione di alimentazione di $ 1700V$ e con $ 50 mV$ di tensione di soglia (valore minimo consentito) si è osservato¸in un tempo di misura di 10s il seguente rapporto tra conteggi:
\begin{equation}
\frac{N-noise}{N-fascio}=\frac{216}{64316}=0.0033
\end{equation}
dell'ordine dell'incertezza statistica sui conteggi in presenza di sorgente. \\
Essendo dunque nel range di tensione di alimentazione suggerito trascurabile la frazione di eventi di rumore, ci si attende che la curva di alimentazione si mantenga crescente saturando eventualmente a tensioni sopra le quali tutti i segnali prodotti nello scintillatore vengono amplificate dal PMT in un segnale sopra la soglia di discriminazione. \\
Un discorso analogo è valido anche per lo scintillatore inorganico, per il quale la saturazione dei conteggi è particolarmente visibile. L'operazione è stata effettuata allineando lo scintillatore con il collimatore del fascio.[INSERIRE IMMAGINE]\\
Al fine di assicurarsi della legittimità degli eventi conteggiati in tale fase è stata effettuata una stima degli eventi attesi sullo scintillatore inorganico nota l'attività della sorgente.
Il calcolo assume un'efficienza di rivelazione dell'apparato coinvolto dell'ordine dell'unità, e si riduce alla stima del numero di fotoni emessi dalla sorgente (assunta isotropa l'emissività di quest'ultima) nella porzione dell'angolo solido individuata dal canale di collimazione, del quale sono noti lunghezza e larghezza. Si ottiene: \\
\begin{equation}
(\frac{dN}{dt})_{est.}\simeq 2.5\cdot 10^{4} evts./s \ \ \ \ (\frac{dN}{dt})_{mis.}=2.75\cdot 10^{4} evts./s
\end{equation}


\subsection{Caratteristiche geometriche del fascio} \label{subsec:geom_fascio}
\'E necessario studiare l'estensione angolare del fascio per stimare correttamente l'errore sulla lettura dell'angolo (sez. \ref{subsec:err_angolo}) e per poter correttamente interpretare i risultati successivi. Per fare ciò si è rimosso lo scintillatore plastico dal sistema e si è studiato l'andamento dei conteggi dello scintillatore inorganico in funzione della sua posizione angolare. Il risultato è mostrato in fig.!!!REF!!!.

%\includegraphics[width\textwidth]{•}

La curva risulta essere ben descritta da una gaussiana, con parametri:
\begin{equation}
\mu = ...°
\nonumber
\end{equation}
\begin{equation}
\sigma = ...°
\nonumber
\end{equation}


L'asimmetria rispetto a $\theta = 0°$ determina la necessità di traslare le future misure angolari di una quantità $\mu$.


\subsection{Stima dell'incertezza sull'angolo} \label{subsec:err_angolo}
La misura della posizione angolare dello scintillatore inorganico è soggetta ad un'incertezza legata sia alla lettura, ovvero alla capacità dell'occhio di individuare il centro dello scintillatore, sia alla distanza tra la proiezione del punto di interazione sul piano del goniometro e il centro del goniometro stesso.
\subsubsection{Incertezza sulla lettura}

Una prima fonte di errore sulla lettura dell'angolo sono rappresentate dalla limitata capacità da parte dell'occhio umano di valutare l'allineamento tra lo scintillatore e l'indicatore dell'angolo sul goniometro. Si è stimata l'incertezza così introdotta in $0.1°$ effettuando più volte operazioni fittizie di allineamento e verificando di volta in volta l'errore angolare commesso.

\subsubsection{Incertezza sul punto di interazione}
Questo tipo di incertezza, come si vedrà in sez.!!!! non si riflette in un effettivo errore sull'angolo, in quanto influisce solamente sulla forma dello spettro in energia osservato e viene dunque incluso nell'errore delle misure di energia.  \\
\vspace{0.8 cm}
Calcoli teorici\cite{compton_total} forniscono il valore della sezione d'urto Compton totale:

\begin{equation}
\sigma = 3.7 \cdot 10^{-25} cm^2
\nonumber
\end{equation}

Considerando l'energia media della sorgente utilizzata (Cobalto, $E/m_e = 2.45$) gli urti possono essere considerati avvenenti tra fotoni ed elettroni liberi. Considerata anche la densità di un polimero tipico (circa $1 g/cm^3$) e il fatto che l'elemento preponderante in questi materiali è il carbonio (in cui metà della massa è dovuta a protoni) è possibile stimare la densità numerica di elettroni, da cui il libero cammino medio di un fotone: \\
 
\begin{equation}
\lambda = \frac{1}{n_{elettroni} \cdot \sigma} = \frac{m_{protone}}{\frac{1}{2} \cdot \rho \cdot \sigma} \approx 10 \ cm
\nonumber
\end{equation}

Essendo questa misura un fattore quasi 10 maggiore dello spessore dello scintillatore plastico, è ragionevole supporre che la probabilità di interazione sia costante lungo lo spessore dello scintillatore. La non conoscenza del punto di interazione porta ad un errore sulla lettura dell'angolo, come è schematizzato in fig.!!!REF!!!. 

%\includegraphics[width=\textwidth]{""} \label{fig:err_angolo}

Si può calcolare:
\begin{equation}
\Delta \alpha_v = \frac{\Delta h_v}{L} sin\alpha
\nonumber
\end{equation}

Il punto di interazione non è determinato anche a causa dell'estensione del fascio stesso (sez. \ref{subsec:geom_fascio}). Questo si traduce in un'incertezza sull'angolo che può essere espressa matematicamente in modo analogo alla precedente:

\begin{equation}
\Delta \alpha_o = \frac{\Delta h_o}{L} cos\alpha
\nonumber
\end{equation}

Tuttavia una rapida stima permette di dimostrare che quest'ultimo produce un effetto trascurabile rispetto al precedente. Come è stato misurato in sez. \ref{subsec:geom_fascio}, il fascio ha una distribuzione ben descritta da una gaussiana con $\sigma = 3.97 °$. Sapendo la distanza tra lo scintillatore inorganico e quello plastico (sez. \ref{sec:apparato}) si trova, con metodi geometrici, che il segmento intercettato dal fascio sullo scintillatore plastico misura \\
\begin{equation}
\Delta h_o \approx 0.5 mm
\end{equation}

da cui si può ricavare \\

da confrontare con il $\Delta \alpha_v$ ottenuto utilizzando $\Delta h_v = 0.72 \ cm$, corrispondente alla deviazione standard di una distribuzione uniforme di ampiezza uguale allo spessore dello scintillatore plastico. Essendo l'errore approssimativamente lineare in $\Delta h$ risulta che: \\
\begin{equation}
\frac{\Delta \alpha_o}{\Delta \alpha_v} = \frac{\Delta h_o}{\Delta h_v} = 0.07 
\end{equation}

e dunque trascurabile.

Un'altra fonte di errore è quella dovuta all'estensione del rivelatore stesso, che occupa una porzione angolare calcolabile, una volta nota la configurazione geometrica, di \\
\begin{equation}
\Delta \alpha = \frac{5.08}{30} rad = 9.7°
\nonumber
\end{equation}





\subsection{Calibrazione dello scintillatore inorganico}

Essendo lo scintillatore inorganico in grado di produrre un segnale di output proporzionale all'energia in esso depositata dai prodotti, è necessaria un'operazione di calibrazione per determinare la legge che associ a ciascun canale dell'analizzatore la corrispondente energia rilasciata nello scintillatore. 
Per fare ciò si è fatto uso di quattro sorgenti di piccola intensità per le quali sono ben note a priori le energie dei fotoni emessi nel decadimento:

[TABELLA CON SORGENTI E PICCHI]

Onde minimizzare la contaminazione da parte dei fotoni emessi dalla sorgente principale, la procedura è stata effettuata ponendo lo scintillatore inorganico ad un angolo di circa $90°$ dalla direttrice del fascio, ponendo inoltre degli appositi schermi in piombo tra la sorgente primaria e il rivelatore. Vista la scarsa intensità delle sorgenti di calibrazione, queste ultime sono state poste direttamente a contatto con il rivestimento in alluminio in modo tale da massimizzare la frazione di angolo solido coperto dal rivelatore.\\
(ERRORI SULLA POSIZIONE)\\
Gli spettri così ottenuti grazie all'analizzatore multicanale sono visualizzati e salvati sotto forma di istogramma, rendendo così possibile l'estrazione della posizione in canali dei picchi mediante un fit gaussiano degli stessi. La legittimità del fit gaussiano deriva dalla monocromaticità delle sorgenti. In seguito tuttavia agli errori sperimentali collegati al processo di rivelazione e al rumore introdotto nella catena di lettura, la distribuzione ideale a delta di Dirac si manifesta come una Gaussiana con una determinata FWHM.\\

Dati i punti disponibili, non è stato osservato un andamento lineare per la curva di calibrazione. Si è pertanto proceduto con un fit polinomiale di grado superiore.\\
Al fine di determinare il grado del migliore fit , si è verificato il più piccolo grado per cui il coefficiente del polinomio risultato del fit è compatibile con zero. Quando ciò accade il coefficiente in questione sarà distribuito come una \cite{delprete_normalreg} Normale a media 0 e varianza ignota $\sigma$ :
\begin{equation}
a_{r}\ \sim \  N(0,\sigma) \ \ \ \rightarrow \ \ \  \dfrac{a^2_{r}}{\sigma^2}\ \sim\ \chi^2_{1} 
\nonumber
\end{equation}
Dove $r$ rappresenta il numero di parametri stimati attraverso il fit polinomiale. Il grado del polinomio utilizzato è pertanto $r-1$. Si procede testando di volta in volta per vari valori di $r$ l'ipotesi nulla di compatibilità con zero del coefficiente del grado massimo. Non essendo nota la varianza sui parametri, ma solo la sua stima prodotta dal fit, il test è da eseguirsi sulla distribuzione Fisher-Snedecor \cite{delprete_normalreg}:
\begin{equation}
f\ = \ \frac{a_r^2}{s^2}\ = \ \frac{\chi^2_{r-1}-\chi^2_r}{\chi^2_r}\ \sim \ F_{1;n-r-1}
\nonumber
\end{equation}
Nella quale $n$ rappresenta il numero di punti sperimentali su cui il fit è effettuato. Tale metodo non permette tuttavia di testare il coefficiente del quinto grado, per il quale il numero di gradi di libertà della distribuzione Fisher-Snedecor si riduce a zero. Ci si è pertanto limitati a effettuare i test d'ipotesi per $R<6$. In tabella sono riportati i valori delle variabili statistiche calcolate al fine di ottenere il P-Value corrispondente a ciascuna ipotesi nulla.\\

\subsubsection{Contributo all'incertezza sull'energia}

L'incertezza sull'energia è ricavata simulando un determinato numero di esperimenti secondo lo schema:
\begin{enumerate}
\item Generazione per ogni pseudoesperimento della posizione in canali dei picchi di emissione delle sorgenti di calibrazione secondo una distribuzione gaussiana con valor medio e varianza uguali ai rispettivi valori misurati sperimentalmente.
\item Fit dei punti così generati utilizzando sempre un polinomio di terzo grado.
\item Canale per canale si è costruito un istogramma dei valori dell'energia ad esso corrispondente ottenuta in ciascuno degli pseudoesperimenti.
\item Estrazione, canale per canale, della media e dello stimatore della varianza della distribuzione così tracciata.
\end{enumerate}

La banda di incertezza sulla curva di calibrazione e le incertezze assolute in funzione dell'energia sono mostrate nelle figure \ref{fig:calib_en} e \ref{fig:err_en}.

\begin{figure}[h!] 
\includegraphics[width=\textwidth]{"calibrazione2"}
\caption{...}
\label{fig:calib_en}
\end{figure}

\begin{figure}[h!] 
\includegraphics[width=\textwidth]{"err_en4"}
\caption{...}
\label{fig:err_en}
\end{figure}


\section{Metodo di misura}
\subsection{Determinazione di $m_e$ dal fit della relazione Energia-Angolo}
La procedura sperimentale consiste semplicemente nell'acquisizione delle misure di spettro da parte dello scintillatore inorganico a diversi angoli di scattering, sfruttando la guida su cui quest'ultimo è montato. La relazione su cui verrà effettuato il fit è:
\begin{equation}
E'\ = \ \frac{E}{1+\frac{E}{m_e}\left(1-\cos{\theta}\right)}
\end{equation}
\subsection{Segnale di gate}
Al fine di ridurre il rapporto segnale/rumore nelle misure di spettro [solo in queste???] si sfrutta la modalità di acquisizione con gate dell'analizzatore multicanale. Tale segnale si costruisce a partire dalla coincidenza tra gli output dei due scintillatori. In tale modo si ordina l'acquisizione del segnale in output dell'inorganico solamente se viene rivelato in coincidenza temporale ad esso un evento nello scintillatore plastico, idealmente dovuto all'elettrone rinculante.
\\
Il segnale in ingresso al gate deve essere conforme per durata al segnale di input all'analizzatore, a sua volta in uscita dal sistema di amplificazione-shaping. Per generare un segnale di questo tipo si è proceduto in questo modo:

\begin{itemize}
	\item Invio dell'output del NaI al modulo di discriminazione. A causa dell'elevata ampiezza e durata dei segnali in uscita dal NaI, il discriminatore ha per output non uno, ma un treno di segnali digitali della durata selezionata sino a quando il segnale analogico non passa sotto soglia.
	\item Il treno di segnali viene mandato come input ad un modulo dual timer che restituisce un unico segnale digitale di durata impostata compatibile con la durata del segnale analogico.
	\item Il segnale così generato viene inviato come input ad una seconda unità di dual timing al fine di ottenere a partire da esso un segnale digitale dalla durata dell'ordine delle decine di nanosecondi in coincidenza con la fase di salita del segnale dell'inorganico.
	\item Il segnale viene inviato al modulo di coincidenza con l'output digitalizzato del plastico.
	\item L'output della coincidenza è inviato in input ad una terza unità dual timer al fine di ottenere un segnale della stessa durata del segnale dello scintillatore inorganico amplificato e formato.
\end{itemize}

L'efficacia del metodo di acquisizione con gate dipende però fortemente dalla rivelabilità dell'elettrone rinculante, e sarà pertanto massimamente efficacie ad angoli di diffusione alti, per i quali l'energia cinetica trasferita è massima. \\
A piccoli angoli di diffusione, per contro, l'efficienza del gate risulta ampiamente ridotta, al punto da rendere la procedura completamente inefficace. [MAGARI IMMAGINE GATE NO GATE STESSO ANGOLO]. Ci si è pertanto limitati ad sfruttare il segnale di gate per angoli di diffusione $|\alpha|>30°$. 

\subsubsection{Caratteristiche degli spettri}
Nelle figure [1 e 2!!!!!!!!!!!!!!!!!!!] sono mostrati gli spettri ottenuti nelle due diverse modalità di acquisizione posizionando lo scintillatore inorganico ad un angolo di 30° sulla destra dello zero del goniometro. \\
Lo spettro acquisito senza l'ausilio del gate presenta due picchi marcati in corrispondenza delle energie dei fotoni primari emessi dalla sorgente. Se ne deduce che per angoli uguali o inferiori a 30° il numero di fotoni che penetra la schermatura della sorgente venendo così rivelata nell'inorganico è superiore al numero di eventi legittimi rivelati dall'inorganico in seguito ad un'interazione Compton nello scintillatore plastico.
La [figura2!!!!!!!!!!!!!!!] mostra come l'acquisizione in modalità GATE permetta di ridurre drasticamente il fondo di fotoni diretti, come deducibile dal rapporto tra le ampiezze tra i fotopicchi (non risolti) e i picchi, appena accennati, di fascio diretto.\\
Nello spettro ottenuto con l'ausilio del gate è inoltre ben distinguibile la \textit{spalla Compton}, dovuta agli eventi Compton ad angolo massimo all'interno dello scintillatore inorganico. Un elemento analogo è ravvisato anche nella [figura1], per quanto in questo caso esso sia dovuto agli eventi di backscattering nell'inorganico causati dai fotoni di fascio diretto, in maggioranza.\\
L'intervallo di energie dei fotoni Compton nel presente esperimento comprende l'energia di soglia sopra la quale è possibile la produzione di coppia $e^+\ e^-$. Ciononostante, alle energie dell'esperimento le sezioni d'urto di produzione e Compton stanno nel seguente rapporto:
\begin{equation}
\frac{\sigma_{pair}}{\sigma_{compt}}\simeq 10^{-2}
\end{equation}
Il che giustifica la non visibilità dei tipici picchi a $E_\gamma \ - \ m_e c^2$ e $E_\gamma \ - \ 2m_e c^2$ dovuti alla fuga di uno o entrambi i fotoni da $511 keV$ prodotti nell'annichilazione del positrone proveniente dall'evento di produzione di coppia.\\
Attenendosi sempre allo spettro con gate [immag!!!!!2], si osserva una struttura complessa a energie di circa $200 keV$. Essa è dovuta al gran numero di fotoni primari che subiscono un'interazione Compton a $\theta \ \simeq \ 180°$ nel materiale di schermo in prossimità della sorgente per poi fuoriuscire dal vano, subire un'interazione Compton nel plastico ed essere rivelati infine nello scintillatore inorganico. L'energia di questi fotoni è circa:

\begin{equation}
E'\ \simeq \ \frac{E}{1+\frac{2E}{m_e}} \ = \ 210\ \emph{keV}
\end{equation}

I fotoni in questione vengono rivelati nello scintillatore inorganico tramite i medesimi processi che interessano i fotoni primari, e risulteranno dunque in una struttura spettrale analoga, con fotopicchi, spalla e continuo Compton difficilmente risolvibili. \\
%RIMUOVERE!!!!!![Il fatto che lo spettro di backscattering si trovi a energie costanti con l'angolo è giustificabile con il fatto che la relazione energia-angolo per un fenomeno compton con energia primaria E generica ha pendenza proporzionale a E quadro. Pertanto la pendenza sarà 0.04 .]RIMUOVEREEEEEEE

A energie di circa $75 \ keV$ si trova inoltre uno stretto picco compatibile con i raggi X \textit{raggi X} molli emessi per fluorescenza dal piombo contenuto nell'apparato di schermo. \cite{X_Ray}

\subsubsection{Risoluzione dei fotopicchi}
La capacità di distinguere due fotopicchi è legata alla differenza di energia tra i picchi dei fotoni diffusi ad un dato angolo. Si è dunque ricavata la relazione algebrica che lega la suddetta differenza alle energie dei fotoni emessi dalla sorgente e all'angolo di misura. 

\begin{figure}
\includegraphics[width=\textwidth]{"distanza_picchi"}
\caption{Differenza di energia tra i picchi dei fotoni diffusi del Cobalto in funzione dell'angolo di misura}
\label{fig:dist_picchi}
\end{figure}

Come si può vedere in Fig. \ref{fig:dist_picchi} la differenza di energia tra i fotopicchi tende a diminuire, rendendo sempre più difficile risolverli all'aumentare dell'angolo di misura. La risolvibilità dipende infatti dal rapporto tra questa quantità e le FWHM dei singoli picchi. Nella maggior parte dei casi non si è riusciti a risolvere i due fotopicchi (sez. \cite{subsec:picchi}), per cui è stato deciso di procedere come segue:

\begin{enumerate}
\item Se la forma dello spettro misurato è significativamente diversa da una gaussiana è stato eseguito un fit \textit{bigaussiano} ed è stata estratta la media aritmetica dei picchi.
\item Se la forma dello spettro misurato è compatibile con quella di una gaussiana è stato estratto il valore del massimo locale.
\end{enumerate}

Ad angoli sufficientemente alti, in cui la differenza di energia tra i picchi è minore, si verifica la seconda condizione. Essendo l'altezza dei due picchi generalmente diversa, è possibile che l'energia del massimo non corrisponda alla media dell'energia dei due picchi. Per studiare quest'effetto si è stimata l'altezza relativa dei picchi ad un determinato angolo supponendo che i due picchi abbiano la stessa FWHM:

\begin{equation}
N(\theta) \propto \frac{d\sigma}{d\Omega} \Phi \rightarrow \frac{N_2(\theta)}{N_1(\theta)} \propto \frac{\frac{d\sigma_2}{d\Omega}}{\frac{d\sigma_1}{d\Omega}} \frac{\Phi_2}{\Phi_1}
\nonumber
\end{equation}



\section{Risultati sperimentali}
\subsection{Misura dei picchi} \label{subsec:picchi}



\section{Conclusioni}
Fare il mea culpa sulla determinazione del punto di lavoro dell'inorganico che si poteva in qualche modo cercare un'alimentazione per la quale era minima l'analisi della FWHM, poi distanza tra scintillatore plastico e sorgente

\input{bibcompton.bib}


\end{document}
