\documentclass[11pt]{article}
\usepackage{makeidx}
\usepackage{graphicx}
\usepackage{amsmath}
\usepackage{amssymb}
\usepackage{latexsym}
\renewcommand\refname{Referenze}
\usepackage[utf8x]{inputenc}
\usepackage{titlesec}
\usepackage{bm}
\usepackage{mathtools}
\usepackage[document]{ragged2e}
\titleformat{\section}{\huge\normalfont\bf}{\thesection.\hspace{5pt}}{5pt}{\vspace{1cm}}
\titleformat*{\subsection}{\Large\bfseries}
\usepackage[inner=3cm,outer=3cm]{geometry}

\makeindex

\begin{document}
\justify
\printindex
\Large{A.a. 2013-2014}
\vspace{10cm}
\begin{center}
\Huge\textbf{Diffusione Compton}
\end{center}

\vspace{2cm}
\begin{flushleft}
\textit{Gruppo \textsc{1}} \\
\medskip
Federico \textsc{Massa} \\ 
Marco \textsc{Montella}
\end{flushleft}



\newpage

\begin{abstract}
\justify
 fsf

\end{abstract}
\bigskip

\section{Introduzione}

fsfs

\section{Apparato sperimentale}
\subsection{Goniometro}
Essendo lo scintillatore inorganico vincolato a muoversi su un arco di circonferenza, risulta di fondamentale importanza la misura della posizione angolare di quest'ultimo rispetto a uno zero che non deve necessariamente coincidere con il centro geometrico del fascio. \\
A questo scopo si è posizionato lo scintillatore plastico a circa $3 cm$ dall'estremità del collimatore e nel punto medio della congiungente degli estremi della guida dell'inorganico. Abbiamo quindi posizionato un goniometro prestando attenzione al fatto che il centro si trovasse sulla verticale del punto di incontro tra l'asse del collimatore e il centro del plastico.

\section{Operazioni preliminari}
fdfd
\subsection{Stima della dose}
s
\subsection{Determinazione del punto di lavoro degli scintillatori}
d
\subsection{Caratteristiche geometriche del fascio}
\'E necessario studiare l'estensione angolare del fascio per stimare correttamente l'errore sulla lettura della posizione angolare (sez. \ref{subsec:err_angolo}) e per poter correttamente interpretare i risultati successivi. Per fare ciò si è rimosso lo scintillatore plastico dal sistema e si è studiato l'andamento dei conteggi dello scintillatore inorganico in funzione della sua posizione angolare. Il risultato è mostrato in fig.!!!REF!!!.

%\includegraphics[width\textwidth]{•}



\subsection{Stima dell'incertezza sull'angolo} \label{subsec:err_angolo}
 \'E legittimo supporre infatti che la maggior parte delle interazioni avvengano in un intorno di questo punto, le cui dimensioni sono determinate dalla larghezza del fascio. \\

%La misura della posizione angolare dello scintillatore inorganico avrà quindi un'incertezza legata sia alla lettura, ovvero alla capacità dell'occhio di individuare il centro dello scintillatore e alla parallasse, sia alla distanza tra la proiezione del punto di interazione sul piano del goniometro e il centro del goniometro stesso.


\subsection{Calibrazione dello scintillatore inorganico}
\

\section{Metodo di misura}
\justify
\subsection{Segnale di gate}

\subsection{A}

\section{Risultati sperimentali}
\justify

\section{Conclusioni}
\justify


\input{bibcompton.bib}


\end{document}
